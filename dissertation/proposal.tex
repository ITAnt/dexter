%%%%%%%%%%%%%%% Updated by MR March 2007 %%%%%%%%%%%%%%%%
\documentclass[12pt]{article}
\usepackage{a4wide}
\usepackage[UKenglish]{babel}
\usepackage[UKenglish]{isodate}
\usepackage[style=footnote-dw]{biblatex}
% Package biblatex: '\bibliography' must be given in preamble.
\bibliography{memoire-bb}

\newcommand{\al}{$<$}
\newcommand{\ar}{$>$}

\parindent 0pt
\parskip 6pt

\begin{document}

\thispagestyle{empty}

\rightline{\large\emph{David Brazdil}}
\medskip
\rightline{\large\emph{Trinity Hall}}
\medskip
\rightline{\large\emph{db538}}

\vfil

\centerline{\large Computer Science Tripos Part II Project Proposal}
\vspace{0.4in}
\centerline{\Large\bf Taint-Based Flow Tracing by}
\vspace{0.1in}
\centerline{\Large\bf Bytecode Instrumentation on Android}
\vspace{0.3in}
\cleanlookdateon
\centerline{\large \emph \today}

\vfil

{\bf Project Originator:} \emph{Dr A. R. Beresford}

%\vspace{0.1in}
%
%{\bf Resources Required:} See attached Project Resource Form

\vspace{0.5in}

{\bf Project Supervisor:} \emph{Dr A. R. Beresford}

\vspace{0.2in}

{\bf Signature:}

\vspace{0.5in}

{\bf Director of Studies:}  \emph{Dr S. W. Moore}

\vspace{0.2in}

{\bf Signature:}

\vspace{0.5in}

{\bf Overseers:} \emph{Prof J. M. Bacon}\ and \emph{Prof R. J. Anderson}

\vspace{0.2in}

{\bf Signatures:} 

\vfil
\eject

\section*{Introduction and Description of the Work}

Android is a popular, open-source Linux-based platform for touchscreen
mobile devices, such as smartphones and tablets. Ever since its unveiling 
in 2007, it has been increasing its market share, running on 68.1\% of 
smartphones sold in the second quarter of 2012 according to a report by 
the International Data Corporation. 
\cite{www.idc.com/getdoc.jsp?containerId=prUS23638712}

Unfortunately, with its growing popularity, the platform became a frequent
target of increasingly sophisticated malware, masquerading as legitimate
software, while leaking sensitive data about the users'. More complex
malicious applications even try to hide their activity by exploiting 
security holes of the underlying operating-system level to bypass the 
protection mechanisms of the platform.

Researchers have been coming up with different approaches to solving
this problem, mostly focusing on porting well-established methods from
the desktop environment to resource-limited devices running Android and
other competing mobile operating systems. These methods range from 
lightweight static analysis of the executable code, all the way to 
enforcing sandboxing by virtualization. 

One such approach is shown in TaintDroid \cite{www.appanalysis.org}, 
project developed by a group of researchers from The Pennsylvania State
University, Duke University and Intel Labs. TaintDroid modifies several
core parts of the Android platform, such as the Dalvik VM, the
Android shared library, or even the file-system, adding support for
runtime labelling (tainting) of sensitive data like phone number, contact 
list or GPS location, and tracing the flow of such data through the 
system. By checking the labels of data that are leaving the system, 
e.g. via the network connection, TaintDroid can warn the users about the
possibility of their data being misused. It is therefore an analytical 
tool providing greater insight into the behaviour of third-party 
applications.


my project will implement an analytical tool capable of tracing the flow
of information inside applications running on top of the Dalvik Virtual
Machine, providing the users with means of checking how third-party 
applications are handling 



Ladder Logic is a simple notation used by engineers for controlling
industrial processes. It can be used describe simple programmable
controllers such as those used in traffic lights, lifts or fire safety
systems. Full details of ladder logic can be found in Part 3 of
standard IEC 1131 (available in the library as V111-26).  This project
is however concerned with a cut down variant of the language and a
detailed understanding of the standard will not be necessary.

The concept of Ladder Logic is drawn from conventional relay-based
logic, where relay \& switch contacts are connected in series/parallel
combinations to control the energisation of further relay coils,
indicator lamps, solenoids etc. The relays may have built-in functions
such as time delay, threshold detection etc., so the resulting circuit
can achieve a reasonable degree of sophistication.

Ladder Logic is so called because the two \emph{uprights\/} of the
ladder are effectively the power supply and ground bus-bars, and each
\emph{rung\/} is (in the simplest form) a sequence of contacts in
series/parallel which will energise a given coil. An example of a
Ladder Diagram:

\begin{verbatim}
    | AUTO_MODE      AUTO_CMD            CMD    |
    +---| |-------------| |------+-------( )----+
    |                            |              |
    | AUTO_MODE      MAN_CMD     |              |
    +---|/|-------------| |------+              |
    |                                           |
\end{verbatim}

Where {\tt --| |--} is a normally-open contact, {\tt --|/|--} is a
normally closed contact, and {\tt --(~)--} is a relay coil.

The above diagram translated into (un-optimised) C language would be:

\begin{verbatim}
    CMD = (AUTO_MODE && AUTO_CMD) || (!AUTO_MODE && MAN_CMD);
\end{verbatim}

or into pseudo assembly language it might be:

\begin{verbatim}
    LOAD    AUTO_MODE
    LOAD    AUTO_CMD
    SYS     $AND
    LOAD    AUTO_MODE
    SYS     $NOT
    LOAD    MAN_CMD
    SYS     $AND
    SYS     $OR
    STORE   CMD
\end{verbatim}

In the full Ladder Logic language, elements are not constrained to
simple boolean operations, but can contain sophisticated arithmetic
functions (e.g.\ auto-tuning PID control loops). The size of the
project can be varied as the project progresses by adding more
elements to the language but this should be done only if time permits.

It is tempting for computer professionals to dismiss Ladder Logic as a
crude language, but it is a salutory thought that that almost every
automated industrial process has one or more programmed logic
controller running Ladder Logic, so we are unknowingly relying on
Ladder Logic to provide many of the mass-produced items we use in our
daily lives!

In conventional programming languages, the separation of the source
text into source lines is simply an aid to human-readability. In a
ladder diagram the vertical and horizontal orientation of the text
elements is crucial, and directly effects the meaning of those
elements. Hence the source code is effectively two-dimensional, as
opposed to the one-dimensional (linear) source code of conventional
languages. One key part of the project is to create a compiler that
can handle such two-dimensional source code in a clean,
well-structured fashion, which allows extra graphical elements to be
added with minimal effort.


\section*{Resources Required}

\al\emph{The most suitable language may turn out to be non-standard and
  its availability needs to be established.  The project can also be
  undertaken (with permission) on a privately owned PC.}\ar


\section*{Starting Point}

\al\emph{This is the place to declare any prior knowledge relevant to
  the project.  For example any relevant courses taken prior to the
  start of the Diploma year.}\ar


\section*{Substance and Structure of the Project}

The objective of the project is to design a ladder-logic style
language and implement its compiler and interpreter/simulator so that
the behaviour of a controller specified in the language can be
simulated.  The input of the compiler should be ASCII text to
represent the two-dimensional logic diagram.
 
Notations for wire joining and crossing should be chosen, and also a
method of entering relays with delayed action should be designed. The
exact rules governing how names can be associated with relays and
coils must be specified. The chosen ladder logic notation should then
be tried out by specifying some simple control circuits. Some of these
will be used later when the system is being tested.

Once the language details have been clarified, work can start on the
compiler.  This could consist of a syntax analyser that constructs an
internal representation of logic diagram, followed by a translator
that converts this into a form suitable for interpretation.

One possible representation of the circuit might be a list of relay
nodes, a list of coil nodes and a list of wire nodes with suitable
inter-node references to represent the circuit. For instance, a relay
node could contain pointers to the activating coil and the two wires
attached to its terminals. The syntax analyser will need some form of
lexical analyser to handle low level input and deal with symbolic
identifiers.

The compiler will contain a translation phase that converts this
internal representation of the circuit into a form that is suitable
for efficient interpretation.  This might consist of a combination of
interpretive code and pre-computed tables.

The interpreter must be capable of simulating the execution of the
specified controller.  It is likely that it should be written in the
style of a discrete event simulator with a time queue of events to be
processed in time order. A typically event might be the opening or
closing of a relay at some specified time in the future. If this is
the next event to happen, time will advance to that moment and the
action will be performed.  This may affect the signals on some wires
which in turn may activate or deactivate some relay coils. This may
result in a cascade of new events being sent to the time queue. During
simulation user readable output should be generated so that the
behaviour of the controller can be clearly understood.

Some ingenuity will be required to find a good way of determining
which coils activations change when a given relay changes state.  If
there are fewer than about 16 relays this can be precomputed in the
form of a table, but for larger circuits other strategies should be
considered. A possible approach might be based on dynamic programming.


\section*{Variants}
 
The entire project can be implemented in Java, C or C++ but other
implementation languages could be considered.

Rather than using an ASCII representation of the logic, it would be
possible to use a graphical representation constructed by a specially
written graphical editor. This may, however, make the project
too large for most candidates.

It may turn out to be a better idea for the simulator to run to
completion generating an internal representation of the behaviour that
can be inspected later by an interactive viewer. The viewer program
should allow the user to decide what parts of the circuit and/or time
period to view.  If this scheme were adopted it would be necessary to
design a good internal representation of the behaviour. Ease of
generation, ease of use and compactness would be the main conflicting
issues affecting the design.  The viewer could generate its output in
simple ASCII text, or it could generate simple pictures in ASCII using
the ``curses'' library package under Unix, or it could generate more
attractive graphical output using, for example, the graphical
facilities of Java.


\section*{Success Criterion}

For the project to be deemed a success the following items must be
successfully completed.

\begin{enumerate}

\item A notation in the style of ladder logic needs to be designed and
  specified in detail.

\item An internal representation of the ladder logic program must be
  designed.

\item Program/programs must be designed and implemented to convert
  this notation into a form suitable for interpretation/simulation.

\item The data structures and algorithms used by the discrete event simulator
must be designed and implemented.

\item Details of what output the system should generate and how it
  should be controlled (by the user) must be specified.

\item Demonstration test programs should be written and run to
  demonstrate the the project works.

\item Some indication of the efficiency of the implementation in terms
  of space and time should be given.

\item The dissertation must be planned and written.

\end{enumerate}

\al\emph{The sketches above are to be taken as starting points for
  investigation of the literature and discussion with your Supervisor.
  A look at the 1996 Diploma Dissertation, {\rm A Ladder Logic
    Compiler and Interpreter}, by Johnson Adesanya, is likely to prove
  useful since it provides some pointers into the literature, and
  shows in detail one way of solving the problem}.\ar

\medskip \al\emph{It would also be possible to adjust the style and
  emphasis of the project either towards the best possible
  computational capability for your code, or in the direction of
  better human/machine interaction, particularly for the visualisation
  of the behaviour of given circuit designs}.\ar



\section*{Timetable and Milestones}

\al\emph{In the following scheme, weeks are numbered so that the week
  starting on the day on which Project Proposals are handed in is
  Week~1.  The year's timetable means that the deadline for submitting
  dissertations is in Week~34.}\ar

\al\emph{In the Project Proposal that you hand in, {\rm actual dates}
  should be used instead of week numbers and you should show how these
  dates relate to the periods in which lectures take place. Week~1
  starts immediately after submission of the Project Proposal.}\ar

\al\emph{The timetable and milestones given below refer to just one
  particular interpretation of this document.  Even if you select
  exactly this interpretation you will need to review the suggested
  timetable and adjust the dates to allow as precisely as you can for
  the amount of programming and other related experience that you have
  at the start of the year.  Take account of the dates you and your
  Supervisor will be working in Cambridge outside Lecture Term.  Note
  that some candidates write the Introduction and Preparation chapters
  of their dissertations quite early in the year, while others will do
  all their writing in one burst near the end}.\ar


\subsection*{Before Proposal submission}

\al\emph{This section will not appear in your Project Proposal.}\ar
 
Submission of Phase~1 Report Form. Discussion with Overseers and
Director of Studies.  Allocation of and discussion with Project
Supervisor, preliminary reading, choice of the variant on the project
and language \al\emph{Java in this example\/}\ar, writing Project
Proposal.  Discussion with Supervisor to arrange a schedule of regular
meetings for obtaining support during the course of the year.

Milestones: Phase~1 Report Form (on the Monday immediately following
the main Briefing Lecture), then a Project Proposal complete with as
realistic a timetable as possible, approval from Overseers and
confirmed availability of any special resources needed. Signatures
from Supervisor and Director of Studies.


\subsection*{Weeks 1 to 5}

\al\emph{Real work on the project starts here (as distinct from just
  work on the proposal).  A significant problem for Diploma candidates
  is that this critical period largely coincides with the Christmas
  vacation.  There is no guarantee that supervisors will be available
  outside Lecture Term, but Diploma students take much less of a
  Christmas break than undergraduates do, and so have some opportunity
  for uninterrupted reading and initial practical work at this stage.
  It is important to have completed some serious work on the project
  before the pressures of the Lent Term become all too apparent.}\ar

Study C and the particular implementation of it to be used.  Practise
writing various small programs, including key fragments of the compiler
and interpreter.

Milestones: Some working example C programs including code to deal
with symbol tables in the lexical analyser, and part implementation of
the time queue to be used in the interpreter/simulator.


\subsection*{Weeks 6 and 7}

Further literature study and discussion with Supervisor to ensure that
the chosen data structures are satisfactory.  Implementation of the
syntax analyser and debugging code to help test it.  This is likely to
be code that can be used to display the data structures in
human-readable form so that it is possible to check that they are as
expected.

Milestones: Ability to construct and display data structures that
represent simple ladder logic programs such as:

\begin{verbatim}
    |    A               B                C     |
    +---| |-------------| |--------------( )----+
\end{verbatim}


\subsection*{Weeks 8 to 10}

Implementation of the translation phase of the compiler. This will of
a decision to be made on what target code to generate. The obvious contnders
are: an interpretive code, Java or C.

Start to plan the Dissertation, thinking ahead especially to the
collection of examples and tests that will be used to demonstrate that
the project has been a success. 

Milestones: Ability to compile some very simple ladder logic programs,
and print out some human readable version of the target code produced.


\subsection*{Weeks 11 and 12}

Complete code for the interpreter/simulator, or the environment in
which to run the generated Java or C target code.

Prepare further test cases.  Review timetable for the remainder of the
project and adjust in the light of experience thus far.  Write the
Progress Report drawing attention to the code already written,
incorporating some examples, and recording any augmentations which at
this stage seem reasonably likely to be incorporated.

Milestones: Simple ladder logic programs should now compile and run
correctly, but probably with some serious inefficiencies in the code.
Progress Report submitted and entire project reviewed both personally
and with Overseers.


\subsection*{Weeks 13 to 19 (including Easter vacation)}

Rework the entire implementation to enhance the
richness of the language it can deal with. \al\emph{It is possible that
the initial implementation has severe restrictions on the number of
relays or coils that can be handled. Such restrictions can be freed at
this stage.}\ar\  Write initial chapters of the Dissertation.

\al\emph{The Easter break from lectures can provide a time to work on a
substantial challenge such as the computation of logarithms, where an
uninterrupted week can allow you to get to grips with a fairly
complicated algorithm.  This is a good time to put in some quiet work
(while your Supervisor is busy on other things) writing the
Preparation and Implementation chapters of the Dissertation.  By this
stage the form of the final implementation should be sufficiently
clear that most of that chapter can be written, even if the code is
incomplete.  Describing clearly what the code will do can often be a
way of sharpening your own understanding of how to implement it.}\ar

Milestones: Preparation chapter of Dissertation complete,
Implementation chapter at least half complete, code can perform a
variety of interesting tasks and should be in a state that in the
worst case it would satisfy the examiners with at most cosmetic
adjustment.


\subsection*{Weeks 20 to 26}

\al\emph{Since your project is, by now, in fairly good shape there is
a chance to use the immediate run-up to exams to attend to small
rationalisations and to implement things that are useful but fairly
straightforward.  It is generally not a good idea to drop all project
work over the revision season; if you do, the code will feel amazingly
unfamiliar when you return to it.  Equally, first priority has to go
to the exams, so do not schedule anything too demanding on the project
front here.  The fact that the Implementation chapter of the
Dissertation is in draft will mean that you should have a very clear
view of the work that remains, and so can schedule it rationally.}\ar

Work on the project will be kept ticking over during this period but
undoubtedly the Easter Term lectures and examination revision will
take priority.


\subsection*{Weeks 27 to 31}

\al\emph{Getting back to work after the examinations and May Week
  calls for discipline.  Setting a timetable can help stiffen your
  resolve!}\ar

Testing and evaluation.  Finish off otherwise ragged parts of the
code.  Write the Introduction chapter and draft the Evaluation and
Conclusions chapters of the Dissertation, complete the Implementation
chapter.

Milestones: Examples and test cases run and results collected,
Dissertation essentially complete, with large sections of it
proof-read by Supervisor and possibly friends and/or Director of
Studies.


\subsection*{Weeks 32 to 33}

Finish Dissertation, preparing diagrams for insertion.  Review whole
project, check the Dissertation, and spend a final few days on
whatever is in greatest need of attention.

\al\emph{In many cases, once a Dissertation is complete (but not
  before) it will become clear where the biggest weakness in the
  entire work is.  In some cases this will be that some feature of the
  code has not been completed or debugged, in other cases it will be
  that more sample output is needed to show the project's capabilities
  on larger test cases.  In yet other cases it will be that the
  Dissertation is not as neatly laid out or well written as would be
  ideal.  There is much to be said for reserving a small amount of
  time right at the end of the project (when your skills are most
  developed) to put in a short but intense burst of work to try to
  improve matters.  Doing this when the Dissertation is already
  complete is good: you have a clearly limited amount of time to work,
  and if your efforts fail you still have something to hand in!  If
  you succeed you may be able to replace that paragraph where you
  apologise for not getting feature X working into a brief note
  observing that you can indeed do X as well as all the other things
  you have talked about.}\ar


\subsection*{Week 34}

\al\emph{Aim to submit the dissertation at least a week before the
  deadline. Be ready to check whether you will be needed for a\/ {\rm
    viva voce} examination}.\ar

Milestone: Submission of Dissertation. 

\printbibliography
\end{document}
